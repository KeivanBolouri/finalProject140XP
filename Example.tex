% Options for packages loaded elsewhere
% Options for packages loaded elsewhere
\PassOptionsToPackage{unicode}{hyperref}
\PassOptionsToPackage{hyphens}{url}
%
\documentclass[
  oneside,
  open=any]{scrbook}
\usepackage{xcolor}
\usepackage{amsmath,amssymb}
\setcounter{secnumdepth}{5}
\usepackage{iftex}
\ifPDFTeX
  \usepackage[T1]{fontenc}
  \usepackage[utf8]{inputenc}
  \usepackage{textcomp} % provide euro and other symbols
\else % if luatex or xetex
  \usepackage{unicode-math} % this also loads fontspec
  \defaultfontfeatures{Scale=MatchLowercase}
  \defaultfontfeatures[\rmfamily]{Ligatures=TeX,Scale=1}
\fi
\usepackage{lmodern}
\ifPDFTeX\else
  % xetex/luatex font selection
\fi
% Use upquote if available, for straight quotes in verbatim environments
\IfFileExists{upquote.sty}{\usepackage{upquote}}{}
\IfFileExists{microtype.sty}{% use microtype if available
  \usepackage[]{microtype}
  \UseMicrotypeSet[protrusion]{basicmath} % disable protrusion for tt fonts
}{}
\makeatletter
\@ifundefined{KOMAClassName}{% if non-KOMA class
  \IfFileExists{parskip.sty}{%
    \usepackage{parskip}
  }{% else
    \setlength{\parindent}{0pt}
    \setlength{\parskip}{6pt plus 2pt minus 1pt}}
}{% if KOMA class
  \KOMAoptions{parskip=half}}
\makeatother
% Make \paragraph and \subparagraph free-standing
\makeatletter
\ifx\paragraph\undefined\else
  \let\oldparagraph\paragraph
  \renewcommand{\paragraph}{
    \@ifstar
      \xxxParagraphStar
      \xxxParagraphNoStar
  }
  \newcommand{\xxxParagraphStar}[1]{\oldparagraph*{#1}\mbox{}}
  \newcommand{\xxxParagraphNoStar}[1]{\oldparagraph{#1}\mbox{}}
\fi
\ifx\subparagraph\undefined\else
  \let\oldsubparagraph\subparagraph
  \renewcommand{\subparagraph}{
    \@ifstar
      \xxxSubParagraphStar
      \xxxSubParagraphNoStar
  }
  \newcommand{\xxxSubParagraphStar}[1]{\oldsubparagraph*{#1}\mbox{}}
  \newcommand{\xxxSubParagraphNoStar}[1]{\oldsubparagraph{#1}\mbox{}}
\fi
\makeatother

\usepackage{color}
\usepackage{fancyvrb}
\newcommand{\VerbBar}{|}
\newcommand{\VERB}{\Verb[commandchars=\\\{\}]}
\DefineVerbatimEnvironment{Highlighting}{Verbatim}{commandchars=\\\{\}}
% Add ',fontsize=\small' for more characters per line
\usepackage{framed}
\definecolor{shadecolor}{RGB}{241,243,245}
\newenvironment{Shaded}{\begin{snugshade}}{\end{snugshade}}
\newcommand{\AlertTok}[1]{\textcolor[rgb]{0.68,0.00,0.00}{#1}}
\newcommand{\AnnotationTok}[1]{\textcolor[rgb]{0.37,0.37,0.37}{#1}}
\newcommand{\AttributeTok}[1]{\textcolor[rgb]{0.40,0.45,0.13}{#1}}
\newcommand{\BaseNTok}[1]{\textcolor[rgb]{0.68,0.00,0.00}{#1}}
\newcommand{\BuiltInTok}[1]{\textcolor[rgb]{0.00,0.23,0.31}{#1}}
\newcommand{\CharTok}[1]{\textcolor[rgb]{0.13,0.47,0.30}{#1}}
\newcommand{\CommentTok}[1]{\textcolor[rgb]{0.37,0.37,0.37}{#1}}
\newcommand{\CommentVarTok}[1]{\textcolor[rgb]{0.37,0.37,0.37}{\textit{#1}}}
\newcommand{\ConstantTok}[1]{\textcolor[rgb]{0.56,0.35,0.01}{#1}}
\newcommand{\ControlFlowTok}[1]{\textcolor[rgb]{0.00,0.23,0.31}{\textbf{#1}}}
\newcommand{\DataTypeTok}[1]{\textcolor[rgb]{0.68,0.00,0.00}{#1}}
\newcommand{\DecValTok}[1]{\textcolor[rgb]{0.68,0.00,0.00}{#1}}
\newcommand{\DocumentationTok}[1]{\textcolor[rgb]{0.37,0.37,0.37}{\textit{#1}}}
\newcommand{\ErrorTok}[1]{\textcolor[rgb]{0.68,0.00,0.00}{#1}}
\newcommand{\ExtensionTok}[1]{\textcolor[rgb]{0.00,0.23,0.31}{#1}}
\newcommand{\FloatTok}[1]{\textcolor[rgb]{0.68,0.00,0.00}{#1}}
\newcommand{\FunctionTok}[1]{\textcolor[rgb]{0.28,0.35,0.67}{#1}}
\newcommand{\ImportTok}[1]{\textcolor[rgb]{0.00,0.46,0.62}{#1}}
\newcommand{\InformationTok}[1]{\textcolor[rgb]{0.37,0.37,0.37}{#1}}
\newcommand{\KeywordTok}[1]{\textcolor[rgb]{0.00,0.23,0.31}{\textbf{#1}}}
\newcommand{\NormalTok}[1]{\textcolor[rgb]{0.00,0.23,0.31}{#1}}
\newcommand{\OperatorTok}[1]{\textcolor[rgb]{0.37,0.37,0.37}{#1}}
\newcommand{\OtherTok}[1]{\textcolor[rgb]{0.00,0.23,0.31}{#1}}
\newcommand{\PreprocessorTok}[1]{\textcolor[rgb]{0.68,0.00,0.00}{#1}}
\newcommand{\RegionMarkerTok}[1]{\textcolor[rgb]{0.00,0.23,0.31}{#1}}
\newcommand{\SpecialCharTok}[1]{\textcolor[rgb]{0.37,0.37,0.37}{#1}}
\newcommand{\SpecialStringTok}[1]{\textcolor[rgb]{0.13,0.47,0.30}{#1}}
\newcommand{\StringTok}[1]{\textcolor[rgb]{0.13,0.47,0.30}{#1}}
\newcommand{\VariableTok}[1]{\textcolor[rgb]{0.07,0.07,0.07}{#1}}
\newcommand{\VerbatimStringTok}[1]{\textcolor[rgb]{0.13,0.47,0.30}{#1}}
\newcommand{\WarningTok}[1]{\textcolor[rgb]{0.37,0.37,0.37}{\textit{#1}}}

\providecommand{\tightlist}{%
  \setlength{\itemsep}{0pt}\setlength{\parskip}{0pt}}\usepackage{longtable,booktabs,array}
\usepackage{calc} % for calculating minipage widths
% Correct order of tables after \paragraph or \subparagraph
\usepackage{etoolbox}
\makeatletter
\patchcmd\longtable{\par}{\if@noskipsec\mbox{}\fi\par}{}{}
\makeatother
% Allow footnotes in longtable head/foot
\IfFileExists{footnotehyper.sty}{\usepackage{footnotehyper}}{\usepackage{footnote}}
\makesavenoteenv{longtable}
\usepackage{graphicx}
\makeatletter
\newsavebox\pandoc@box
\newcommand*\pandocbounded[1]{% scales image to fit in text height/width
  \sbox\pandoc@box{#1}%
  \Gscale@div\@tempa{\textheight}{\dimexpr\ht\pandoc@box+\dp\pandoc@box\relax}%
  \Gscale@div\@tempb{\linewidth}{\wd\pandoc@box}%
  \ifdim\@tempb\p@<\@tempa\p@\let\@tempa\@tempb\fi% select the smaller of both
  \ifdim\@tempa\p@<\p@\scalebox{\@tempa}{\usebox\pandoc@box}%
  \else\usebox{\pandoc@box}%
  \fi%
}
% Set default figure placement to htbp
\def\fps@figure{htbp}
\makeatother

\makeatletter
\@ifpackageloaded{caption}{}{\usepackage{caption}}
\AtBeginDocument{%
\ifdefined\contentsname
  \renewcommand*\contentsname{Table of contents}
\else
  \newcommand\contentsname{Table of contents}
\fi
\ifdefined\listfigurename
  \renewcommand*\listfigurename{List of Figures}
\else
  \newcommand\listfigurename{List of Figures}
\fi
\ifdefined\listtablename
  \renewcommand*\listtablename{List of Tables}
\else
  \newcommand\listtablename{List of Tables}
\fi
\ifdefined\figurename
  \renewcommand*\figurename{Figure}
\else
  \newcommand\figurename{Figure}
\fi
\ifdefined\tablename
  \renewcommand*\tablename{Table}
\else
  \newcommand\tablename{Table}
\fi
}
\@ifpackageloaded{float}{}{\usepackage{float}}
\floatstyle{ruled}
\@ifundefined{c@chapter}{\newfloat{codelisting}{h}{lop}}{\newfloat{codelisting}{h}{lop}[chapter]}
\floatname{codelisting}{Listing}
\newcommand*\listoflistings{\listof{codelisting}{List of Listings}}
\makeatother
\makeatletter
\makeatother
\makeatletter
\@ifpackageloaded{caption}{}{\usepackage{caption}}
\@ifpackageloaded{subcaption}{}{\usepackage{subcaption}}
\makeatother

\usepackage{hyphenat}
\usepackage{ifthen}
\usepackage{calc}
\usepackage{calculator}

\usepackage{graphicx}
\usepackage{wallpaper}

\usepackage{geometry}

\usepackage{graphicx}
\usepackage{geometry}
\usepackage{afterpage}
\usepackage{tikz}
\usetikzlibrary{calc}
\usetikzlibrary{fadings}
\usepackage[pagecolor=none]{pagecolor}


% Set the titlepage font families







% Set the coverpage font families

\usepackage{bookmark}
\IfFileExists{xurl.sty}{\usepackage{xurl}}{} % add URL line breaks if available
\urlstyle{same}
\hypersetup{
  pdftitle={Civilian Impact of U.S. Drone vs.~Non-Drone Strikes in Somalia and Yemen},
  pdfauthor={Shanmei Wanyan; Daniel Dai; Keivan Bolouri; Itaru Fukushima; Linxue Guo; Evelyn Isaka},
  hidelinks,
  pdfcreator={LaTeX via pandoc}}


\title{Civilian Impact of U.S. Drone vs.~Non-Drone Strikes in Somalia
and Yemen}
\usepackage{etoolbox}
\makeatletter
\providecommand{\subtitle}[1]{% add subtitle to \maketitle
  \apptocmd{\@title}{\par {\large #1 \par}}{}{}
}
\makeatother
\subtitle{Assessing Humanitarian Impact with Count Regression Models.}
\author{Shanmei Wanyan \and Daniel Dai \and Keivan Bolouri \and Itaru
Fukushima \and Linxue Guo \and Evelyn Isaka}
\date{}
\begin{document}
%%%%% begin titlepage extension code

  \begin{frontmatter}

\begin{titlepage}

%%% TITLE PAGE START

% Set up alignment commands
%Page
\newcommand{\titlepagepagealign}{
\ifthenelse{\equal{left}{right}}{\raggedleft}{}
\ifthenelse{\equal{left}{center}}{\centering}{}
\ifthenelse{\equal{left}{left}}{\raggedright}{}
}


\newcommand{\titleandsubtitle}{
% Title and subtitle
{{\large{\bfseries{\nohyphens{Civilian Impact of U.S. Drone
vs.~Non-Drone Strikes in Somalia and Yemen}}}}\par
}%

\vspace{\betweentitlesubtitle}
{
{\large{\textit{\nohyphens{Assessing Humanitarian Impact with Count
Regression Models.}}}}\par
}}
\newcommand{\titlepagetitleblock}{
\titleandsubtitle
}

\newcommand{\authorstyle}[1]{{\large{#1}}}

\newcommand{\affiliationstyle}[1]{{\large{#1}}}

\newcommand{\titlepageauthorblock}{
{\authorstyle{\nohyphens{Shanmei
Wanyan}{\textsuperscript{1}},  \nohyphens{Daniel
Dai}{\textsuperscript{1}},  \nohyphens{Keivan
Bolouri}{\textsuperscript{1}},  \nohyphens{Itaru
Fukushima}{\textsuperscript{1}},  \nohyphens{Linxue
Guo}{\textsuperscript{1}} and \nohyphens{Evelyn
Isaka}{\textsuperscript{1}}}}}

\newcommand{\titlepageaffiliationblock}{
\hangindent=1em
\hangafter=1
{\affiliationstyle{
{1}.~University of California, Los Angeles,~Department of Statistics and
Data Science,~8125 Math Sciences Bldg, Los Angeles, CA 90095


\vspace{1\baselineskip} 
}}
}
\newcommand{\headerstyled}{%
{The Publisher}
}
\newcommand{\footerstyled}{%
{\large{NOAA Fisheries OpenSci\\
Tools for Open Science\\
\url{https://keivanbolouri.github.io/finalProject140XP/}\strut \\}}
}
\newcommand{\datestyled}{%
{}
}


\newcommand{\titlepageheaderblock}{\headerstyled}

\newcommand{\titlepagefooterblock}{
\footerstyled
}

\newcommand{\titlepagedateblock}{
\datestyled
}

%set up blocks so user can specify order
\newcommand{\titleblock}{\newlength{\betweentitlesubtitle}
\setlength{\betweentitlesubtitle}{\baselineskip}
{

{\titlepagetitleblock}
}

\vspace{4\baselineskip}
}

\newcommand{\authorblock}{{\titlepageauthorblock}

\vspace{2\baselineskip}
}

\newcommand{\affiliationblock}{{\titlepageaffiliationblock}

\vspace{1pt}
}

\newcommand{\logoblock}{{\includegraphics[width=0.25\textheight]{img/logo.png}}

\vspace{2\baselineskip}
}

\newcommand{\footerblock}{{\titlepagefooterblock}

\vspace{1pt}
}

\newcommand{\dateblock}{}

\newcommand{\headerblock}{{\titlepageheaderblock

\vspace{0pt}
}}
\newgeometry{top=3in,bottom=1in,right=1in,left=1in}
% background image
\newlength{\bgimagesize}
\setlength{\bgimagesize}{0.5\paperwidth}
\LENGTHDIVIDE{\bgimagesize}{\paperwidth}{\theRatio} % from calculator pkg
\ThisULCornerWallPaper{\theRatio}{img/corner-bg.png}

\thispagestyle{empty} % no page numbers on titlepages


\newcommand{\vrulecode}{\textcolor{black}{\rule{\vrulewidth}{\textheight}}}
\newlength{\vrulewidth}
\setlength{\vrulewidth}{1pt}
\newlength{\B}
\setlength{\B}{\ifdim\vrulewidth > 0pt 0.05\textwidth\else 0pt\fi}
\newlength{\minipagewidth}
\ifthenelse{\equal{left}{left} \OR \equal{left}{right} }
{% True case
\setlength{\minipagewidth}{\textwidth - \vrulewidth - \B - 0.1\textwidth}
}{
\setlength{\minipagewidth}{\textwidth - 2\vrulewidth - 2\B - 0.1\textwidth}
}
\ifthenelse{\equal{left}{left} \OR \equal{left}{leftright}}
{% True case
\raggedleft % needed for the minipage to work
\vrulecode
\hspace{\B}
}{%
\raggedright % else it is right only and width is not 0
}
% [position of box][box height][inner position]{width}
% [s] means stretch out vertically; assuming there is a vfill
\begin{minipage}[b][\textheight][s]{\minipagewidth}
\titlepagepagealign
\titleblock

\authorblock

\affiliationblock

\vfill

\logoblock

\footerblock
\par

\end{minipage}\ifthenelse{\equal{left}{right} \OR \equal{left}{leftright} }{
\hspace{\B}
\vrulecode}{}
\clearpage
\restoregeometry
%%% TITLE PAGE END
\end{titlepage}
\setcounter{page}{1}
\end{frontmatter}

%%%%% end titlepage extension code

\renewcommand*\contentsname{Table of contents}
{
\setcounter{tocdepth}{2}
\tableofcontents
}
\listoffigures
\listoftables

\mainmatter
\chapter{Abstract}\label{abstract}

Since 2002, the United States has conducted largely hidden
counterterrorism campaigns in countries such as Somalia and Yemen,
raising ongoing concerns about their humanitarian impact. This project
asks how the characteristics and civilian costs of U.S. strikes differ
between these two theaters of war. Using open-source strike records
compiled by independent monitoring organizations, including the Bureau
of Investigative Journalism, We construct a combined dataset of U.S.
actions in Somalia and Yemen and analyze casualty patterns with negative
binomial regression. The analysis tests three hypotheses: whether
civilian casualty rates differ by country, whether drone strikes have
different effects across countries, and whether reporting uncertainty
varies between regions. The results show that, controlling for strike
characteristics and total fatalities, strikes in Yemen are associated
with nearly five times the civilian casualties of strikes in Somalia. By
contrast, there is no evidence that the impact of drone strikes on
civilian harm differs between the two countries, nor that overall
reporting uncertainty is systematically higher in one region than the
other. However, uncertainty is greater for drone and unconfirmed strikes
and lower when U.S. involvement is confirmed. These findings underscore
the unequal humanitarian burdens across theaters of U.S.
counterterrorism and highlight the need for more transparent and
consistent casualty reporting.

\chapter{Introduction}\label{introduction}

\section{}\label{section}

Since 2002, the United States has waged a largely clandestine drone war
in countries such as Yemen and Somalia, often far from public scrutiny
{[}1{]}. Although these counterterrorism strikes aim to eliminate
militant targets while minimizing risk to U.S. personnel, their
humanitarian consequences remain a pressing concern. The cost to
civilian life can be substantial. For example, an investigation found
that roughly one-third of those killed by U.S. drone strikes in Yemen in
2018 were likely civilians or pro-government allies {[}2{]}.

This research addresses a central question: \textbf{Do the
characteristics and human costs of U.S. counterterrorism strikes differ
between Somalia and Yemen---and if so, how?} Understanding such
differences is important both theoretically and practically.\\
Theoretically, comparing two distinct drone theaters can reveal how
local conditions---such as insurgent behavior, intelligence quality, and
terrain---shape strike outcomes. Practically, identifying where drone
operations are less effective in sparing civilians can guide
improvements in targeting procedures, transparency, and accountability
mechanisms.

These operations are often carried out ``out of sight,'' yet their
humanitarian consequences are very real {[}1{]}. Official reporting has
historically underestimated civilian casualties, prompting independent
organizations to investigate and publish alternative estimates {[}2{]}.
For instance, the U.S. government once claimed only 64--116 civilian
deaths from drone strikes outside declared warzones between 2009 and
2015, whereas independent monitors estimated several times more {[}2{]}.

In response to these discrepancies, numerous efforts have emerged to
document the drone war's toll. Pitch Interactive's \emph{Out of Sight,
Out of Mind} visualization cataloged CIA drone strikes and casualties in
Pakistan {[}1{]}. The Economist released infographics demonstrating
large gaps between official and independent casualty estimates. UCLA's
\textbf{Drone Wars} project created a cross-country dataset covering
Afghanistan, Pakistan, Somalia, and Yemen using records from the Bureau
of Investigative Journalism (BIJ) {[}3,4{]}.

These initiatives highlight the need for \textbf{rigorous, comparative
analysis}. Yet no study has systematically compared Somalia and Yemen
with respect to strike characteristics and humanitarian outcomes. This
paper fills that gap by leveraging detailed open-source strike records
from both countries to quantitatively assess differences in civilian
harm.

We explicitly test three hypotheses:

\begin{enumerate}
\def\labelenumi{\arabic{enumi}.}
\item
  \textbf{Hypothesis 1 -- Civilian Harm Difference:}\\
  Somalia and Yemen differ in their civilian casualty rates.
\item
  \textbf{Hypothesis 2 -- Drone Effectiveness Across Countries:}\\
  The effect of drone strikes on civilian casualties differs between
  Somalia and Yemen.
\item
  \textbf{Hypothesis 3 -- Reporting Uncertainty:}\\
  Casualty reporting uncertainty differs between regions.
\end{enumerate}

To evaluate these hypotheses, we construct a comprehensive dataset of
U.S. counterterrorism strikes in Somalia and Yemen from independent
monitoring organizations such as BIJ {[}1{]}. Because fatality reporting
is often uncertain, we use minimum--maximum casualty ranges {[}1{]}.
Civilian casualty counts exhibit strong overdispersion, so we employ
negative binomial regression to estimate the effects of region and
strike characteristics. This modeling framework allows us to determine
whether ``country'' remains a significant predictor of civilian harm
once contextual factors are controlled for.

\chapter{Literature Review}\label{literature-review}

Researchers and monitoring groups have spent many years examining how
many people are killed in U.S. drone strikes, but most work focuses on
one country at a time rather than comparing Somalia and Yemen directly.

Columbia Law School's Human Rights Clinic, in \emph{Counting Drone
Strike Deaths}, shows that official U.S. numbers often underestimate
civilian deaths. They recommend using casualty ranges (minimum--maximum)
because information from the ground is often unclear \textbf{{[}3{]}}.

The Bureau of Investigative Journalism (BIJ) collected open-source
reports for every known strike in Yemen, Somalia, Pakistan, and
Afghanistan. Their database records both minimum and maximum death
counts and distinguishes civilians from militants when possible, noting
that reports are often uncertain or contradictory \textbf{{[}5{]}}.

New America's \emph{Counterterrorism Wars} project compiles strike data
from Yemen and Somalia, listing total strikes and casualty ranges and
explaining how they classify victims when reports are vague or disputed
\textbf{{[}6{]}}.

Together, these sources show that:

\begin{enumerate}
\def\labelenumi{\arabic{enumi}.}
\item
  Independent groups usually find \textbf{more civilian deaths} than
  official U.S. reports.
\item
  Although detailed data exist for Yemen and Somalia, most previous
  analyses summarize each country separately rather than compare them
  statistically.
\end{enumerate}

Our study fits into this work by using open-source strike records to
conduct a direct, quantitative comparison between Somalia and Yemen.
Using negative binomial regression, we test whether the countries differ
in civilian casualty rates and the uncertainty of reported casualties,
controlling for strike characteristics.

\chapter{Methods}\label{methods}

\section{Statistical Methods}\label{statistical-methods}

\subsection{Hypothesis Tests}\label{hypothesis-tests}

\begin{center}\rule{0.5\linewidth}{0.5pt}\end{center}

\subsection{\texorpdfstring{\textbf{Hypothesis 1: Civilian Harm
Difference}}{Hypothesis 1: Civilian Harm Difference}}\label{hypothesis-1-civilian-harm-difference}

\[
\begin{aligned}
H_{0}: &\ \text{Drone strikes have the same civilian impact in Somalia and Yemen.} \\
H_{1}: &\ \text{Drone strikes have different civilian impacts across the two regions.}
\end{aligned}
\]

To test whether Somalia and Yemen differ in civilian casualty outcomes,
we estimate the model:

\[
\begin{aligned}
E(\text{civilian casualties}) = {} &
\beta_0 
+ \beta_1(\text{region}) 
+ \beta_2(\text{drone}) 
+ \beta_3(\text{US confirmed}) 
+ \beta_4(\text{minimum strikes}) \\
& + \beta_5(\text{total killed})
\end{aligned}
\]

\begin{longtable}[]{@{}
  >{\raggedright\arraybackslash}p{(\linewidth - 2\tabcolsep) * \real{0.2222}}
  >{\raggedright\arraybackslash}p{(\linewidth - 2\tabcolsep) * \real{0.7778}}@{}}
\caption{Variable Definitions}\tabularnewline
\toprule\noalign{}
\begin{minipage}[b]{\linewidth}\raggedright
Variable
\end{minipage} & \begin{minipage}[b]{\linewidth}\raggedright
Description
\end{minipage} \\
\midrule\noalign{}
\endfirsthead
\toprule\noalign{}
\begin{minipage}[b]{\linewidth}\raggedright
Variable
\end{minipage} & \begin{minipage}[b]{\linewidth}\raggedright
Description
\end{minipage} \\
\midrule\noalign{}
\endhead
\bottomrule\noalign{}
\endlastfoot
Civilian casualties & Number of civilians reported killed in the strike
(outcome variable). \\
Region & Country where the strike occurred (Somalia or Yemen). \\
Drone & Indicates whether the strike was carried out by a drone (1 =
drone). \\
US confirmed & Whether the strike was officially confirmed by the U.S.
government. \\
Minimum strikes & Minimum number of strike events associated with the
record. \\
Total killed & Minimum number of total fatalities (civilians +
militants). \\
\end{longtable}

\subsection{\texorpdfstring{\textbf{Hypothesis 2: Drone Effectiveness
Across
Countries}}{Hypothesis 2: Drone Effectiveness Across Countries}}\label{hypothesis-2-drone-effectiveness-across-countries}

\[
\begin{aligned}
H_{0}: &\ \text{Drone use affects civilian casualties in the same way in both Somalia and Yemen.} \\
H_{1}: &\ \text{Drone use affects civilian casualties differently across Somalia and Yemen.}
\end{aligned}
\]

To evaluate whether the civilian impact of drone strikes varies by
region, we include an interaction term between drone use and region:

\[
\begin{aligned}
E(\text{civilian casualties}) = {} &
\beta_0 
+ \beta_1(\text{drone})
+ \beta_2(\text{region}) 
+ \beta_3(\text{drone} \times \text{region}) \\
& + \beta_4(\text{US confirmed})
+ \beta_5(\text{minimum strikes})
+ \beta_6(\text{total killed})
\end{aligned}
\]

\subsection{\texorpdfstring{\textbf{Hypothesis 3: Reporting Uncertainty
Difference}}{Hypothesis 3: Reporting Uncertainty Difference}}\label{hypothesis-3-reporting-uncertainty-difference}

\textbf{H0:} Reporting uncertainty does not differ between Somalia and
Yemen.\\
\textbf{H1:} Reporting uncertainty differs between Somalia and Yemen.

\[
\begin{aligned}
H_{0}: &\ \text{Reporting uncertainty does not differ between Somalia and Yemen.} \\
H_{1}: &\ \text{Reporting uncertainty differs between Somalia and Yemen.}
\end{aligned}
\]

To assess whether casualty reporting uncertainty differs between
regions, we model the uncertainty metric. We modeled casualty reporting
uncertainty (defined as \texttt{max\_killed\ -\ min\_killed}) region and
strike characteristics as predictors.

\[
E(\text{Uncertainty in casualties}) =
\beta_0
+ \beta_1\,\text{Region}
+ \beta_2\,\text{Drone}
+ \beta_3\,\text{US confirmed}
+ \beta_4\,\text{Minimum strikes}
\]

where:

\[
\text{Uncertainty in casualties} = 
\text{Maximum killed} - \text{Minimum killed}
\]

\chapter{Model Selection}\label{model-selection}

Because our prediction variable is a count---specifically, the number of
civilians killed in each strike---we use statistical models designed for
count data. A natural starting point is the \textbf{Poisson regression},
which assumes that the mean and variance of the outcome are equal
\(E(x) = \mathrm{Var}(x)\). However, in our dataset the variance is much
larger than the mean, a condition known as \textbf{overdispersion}. When
overdispersion is present, Poisson regression underestimates the true
variability and produces misleadingly small standard errors. To address
this, we use a \textbf{negative binomial regression}, which adds a
dispersion parameter that allows the variance to exceed the mean. This
makes the negative binomial model much better suited for modeling
drone-strike casualty counts and provides more reliable estimates of how
factors such as region, drone use, and confirmation status relate to
civilian harm.

\begin{Shaded}
\begin{Highlighting}[]
\NormalTok{mean\_civ }\OtherTok{\textless{}{-}} \FunctionTok{mean}\NormalTok{(combined\_model}\SpecialCharTok{$}\NormalTok{civilian\_casualties)}
\NormalTok{var\_civ  }\OtherTok{\textless{}{-}} \FunctionTok{var}\NormalTok{(combined\_model}\SpecialCharTok{$}\NormalTok{civilian\_casualties)}

\FunctionTok{c}\NormalTok{(}\AttributeTok{mean =}\NormalTok{ mean\_civ, }\AttributeTok{variance =}\NormalTok{ var\_civ)}
\end{Highlighting}
\end{Shaded}

\begin{verbatim}
     mean  variance 
0.4338521 8.0043879 
\end{verbatim}

\chapter{Author Contributions}\label{author-contributions}

Author1 designed the research. Author2 carried out all simulations,
analyzed the data. Author1 and Author2 wrote the article.

\chapter{Acknowledgments}\label{acknowledgments}

We thank G. Harrison, B. Harper, and J. Doe for their help.

\chapter{References}\label{references}

\phantomsection\label{refs}


\backmatter


\end{document}
