% Options for packages loaded elsewhere
% Options for packages loaded elsewhere
\PassOptionsToPackage{unicode}{hyperref}
\PassOptionsToPackage{hyphens}{url}
%
\documentclass[
  oneside,
  open=any]{scrbook}
\usepackage{xcolor}
\usepackage{amsmath,amssymb}
\setcounter{secnumdepth}{5}
\usepackage{iftex}
\ifPDFTeX
  \usepackage[T1]{fontenc}
  \usepackage[utf8]{inputenc}
  \usepackage{textcomp} % provide euro and other symbols
\else % if luatex or xetex
  \usepackage{unicode-math} % this also loads fontspec
  \defaultfontfeatures{Scale=MatchLowercase}
  \defaultfontfeatures[\rmfamily]{Ligatures=TeX,Scale=1}
\fi
\usepackage{lmodern}
\ifPDFTeX\else
  % xetex/luatex font selection
\fi
% Use upquote if available, for straight quotes in verbatim environments
\IfFileExists{upquote.sty}{\usepackage{upquote}}{}
\IfFileExists{microtype.sty}{% use microtype if available
  \usepackage[]{microtype}
  \UseMicrotypeSet[protrusion]{basicmath} % disable protrusion for tt fonts
}{}
\makeatletter
\@ifundefined{KOMAClassName}{% if non-KOMA class
  \IfFileExists{parskip.sty}{%
    \usepackage{parskip}
  }{% else
    \setlength{\parindent}{0pt}
    \setlength{\parskip}{6pt plus 2pt minus 1pt}}
}{% if KOMA class
  \KOMAoptions{parskip=half}}
\makeatother
% Make \paragraph and \subparagraph free-standing
\makeatletter
\ifx\paragraph\undefined\else
  \let\oldparagraph\paragraph
  \renewcommand{\paragraph}{
    \@ifstar
      \xxxParagraphStar
      \xxxParagraphNoStar
  }
  \newcommand{\xxxParagraphStar}[1]{\oldparagraph*{#1}\mbox{}}
  \newcommand{\xxxParagraphNoStar}[1]{\oldparagraph{#1}\mbox{}}
\fi
\ifx\subparagraph\undefined\else
  \let\oldsubparagraph\subparagraph
  \renewcommand{\subparagraph}{
    \@ifstar
      \xxxSubParagraphStar
      \xxxSubParagraphNoStar
  }
  \newcommand{\xxxSubParagraphStar}[1]{\oldsubparagraph*{#1}\mbox{}}
  \newcommand{\xxxSubParagraphNoStar}[1]{\oldsubparagraph{#1}\mbox{}}
\fi
\makeatother

\usepackage{color}
\usepackage{fancyvrb}
\newcommand{\VerbBar}{|}
\newcommand{\VERB}{\Verb[commandchars=\\\{\}]}
\DefineVerbatimEnvironment{Highlighting}{Verbatim}{commandchars=\\\{\}}
% Add ',fontsize=\small' for more characters per line
\usepackage{framed}
\definecolor{shadecolor}{RGB}{241,243,245}
\newenvironment{Shaded}{\begin{snugshade}}{\end{snugshade}}
\newcommand{\AlertTok}[1]{\textcolor[rgb]{0.68,0.00,0.00}{#1}}
\newcommand{\AnnotationTok}[1]{\textcolor[rgb]{0.37,0.37,0.37}{#1}}
\newcommand{\AttributeTok}[1]{\textcolor[rgb]{0.40,0.45,0.13}{#1}}
\newcommand{\BaseNTok}[1]{\textcolor[rgb]{0.68,0.00,0.00}{#1}}
\newcommand{\BuiltInTok}[1]{\textcolor[rgb]{0.00,0.23,0.31}{#1}}
\newcommand{\CharTok}[1]{\textcolor[rgb]{0.13,0.47,0.30}{#1}}
\newcommand{\CommentTok}[1]{\textcolor[rgb]{0.37,0.37,0.37}{#1}}
\newcommand{\CommentVarTok}[1]{\textcolor[rgb]{0.37,0.37,0.37}{\textit{#1}}}
\newcommand{\ConstantTok}[1]{\textcolor[rgb]{0.56,0.35,0.01}{#1}}
\newcommand{\ControlFlowTok}[1]{\textcolor[rgb]{0.00,0.23,0.31}{\textbf{#1}}}
\newcommand{\DataTypeTok}[1]{\textcolor[rgb]{0.68,0.00,0.00}{#1}}
\newcommand{\DecValTok}[1]{\textcolor[rgb]{0.68,0.00,0.00}{#1}}
\newcommand{\DocumentationTok}[1]{\textcolor[rgb]{0.37,0.37,0.37}{\textit{#1}}}
\newcommand{\ErrorTok}[1]{\textcolor[rgb]{0.68,0.00,0.00}{#1}}
\newcommand{\ExtensionTok}[1]{\textcolor[rgb]{0.00,0.23,0.31}{#1}}
\newcommand{\FloatTok}[1]{\textcolor[rgb]{0.68,0.00,0.00}{#1}}
\newcommand{\FunctionTok}[1]{\textcolor[rgb]{0.28,0.35,0.67}{#1}}
\newcommand{\ImportTok}[1]{\textcolor[rgb]{0.00,0.46,0.62}{#1}}
\newcommand{\InformationTok}[1]{\textcolor[rgb]{0.37,0.37,0.37}{#1}}
\newcommand{\KeywordTok}[1]{\textcolor[rgb]{0.00,0.23,0.31}{\textbf{#1}}}
\newcommand{\NormalTok}[1]{\textcolor[rgb]{0.00,0.23,0.31}{#1}}
\newcommand{\OperatorTok}[1]{\textcolor[rgb]{0.37,0.37,0.37}{#1}}
\newcommand{\OtherTok}[1]{\textcolor[rgb]{0.00,0.23,0.31}{#1}}
\newcommand{\PreprocessorTok}[1]{\textcolor[rgb]{0.68,0.00,0.00}{#1}}
\newcommand{\RegionMarkerTok}[1]{\textcolor[rgb]{0.00,0.23,0.31}{#1}}
\newcommand{\SpecialCharTok}[1]{\textcolor[rgb]{0.37,0.37,0.37}{#1}}
\newcommand{\SpecialStringTok}[1]{\textcolor[rgb]{0.13,0.47,0.30}{#1}}
\newcommand{\StringTok}[1]{\textcolor[rgb]{0.13,0.47,0.30}{#1}}
\newcommand{\VariableTok}[1]{\textcolor[rgb]{0.07,0.07,0.07}{#1}}
\newcommand{\VerbatimStringTok}[1]{\textcolor[rgb]{0.13,0.47,0.30}{#1}}
\newcommand{\WarningTok}[1]{\textcolor[rgb]{0.37,0.37,0.37}{\textit{#1}}}

\providecommand{\tightlist}{%
  \setlength{\itemsep}{0pt}\setlength{\parskip}{0pt}}\usepackage{longtable,booktabs,array}
\usepackage{calc} % for calculating minipage widths
% Correct order of tables after \paragraph or \subparagraph
\usepackage{etoolbox}
\makeatletter
\patchcmd\longtable{\par}{\if@noskipsec\mbox{}\fi\par}{}{}
\makeatother
% Allow footnotes in longtable head/foot
\IfFileExists{footnotehyper.sty}{\usepackage{footnotehyper}}{\usepackage{footnote}}
\makesavenoteenv{longtable}
\usepackage{graphicx}
\makeatletter
\newsavebox\pandoc@box
\newcommand*\pandocbounded[1]{% scales image to fit in text height/width
  \sbox\pandoc@box{#1}%
  \Gscale@div\@tempa{\textheight}{\dimexpr\ht\pandoc@box+\dp\pandoc@box\relax}%
  \Gscale@div\@tempb{\linewidth}{\wd\pandoc@box}%
  \ifdim\@tempb\p@<\@tempa\p@\let\@tempa\@tempb\fi% select the smaller of both
  \ifdim\@tempa\p@<\p@\scalebox{\@tempa}{\usebox\pandoc@box}%
  \else\usebox{\pandoc@box}%
  \fi%
}
% Set default figure placement to htbp
\def\fps@figure{htbp}
\makeatother

\usepackage{titling}
\pretitle{\vspace*{-3cm}\begin{center}\bfseries\Huge}
\posttitle{\end{center}}
\predate{\begin{center}\large}
\postdate{\end{center}}
\makeatletter
\@ifpackageloaded{caption}{}{\usepackage{caption}}
\AtBeginDocument{%
\ifdefined\contentsname
  \renewcommand*\contentsname{Table of contents}
\else
  \newcommand\contentsname{Table of contents}
\fi
\ifdefined\listfigurename
  \renewcommand*\listfigurename{List of Figures}
\else
  \newcommand\listfigurename{List of Figures}
\fi
\ifdefined\listtablename
  \renewcommand*\listtablename{List of Tables}
\else
  \newcommand\listtablename{List of Tables}
\fi
\ifdefined\figurename
  \renewcommand*\figurename{Figure}
\else
  \newcommand\figurename{Figure}
\fi
\ifdefined\tablename
  \renewcommand*\tablename{Table}
\else
  \newcommand\tablename{Table}
\fi
}
\@ifpackageloaded{float}{}{\usepackage{float}}
\floatstyle{ruled}
\@ifundefined{c@chapter}{\newfloat{codelisting}{h}{lop}}{\newfloat{codelisting}{h}{lop}[chapter]}
\floatname{codelisting}{Listing}
\newcommand*\listoflistings{\listof{codelisting}{List of Listings}}
\makeatother
\makeatletter
\makeatother
\makeatletter
\@ifpackageloaded{caption}{}{\usepackage{caption}}
\@ifpackageloaded{subcaption}{}{\usepackage{subcaption}}
\makeatother

\usepackage{hyphenat}
\usepackage{ifthen}
\usepackage{calc}
\usepackage{calculator}



\usepackage{graphicx}
\usepackage{geometry}
\usepackage{afterpage}
\usepackage{tikz}
\usetikzlibrary{calc}
\usetikzlibrary{fadings}
\usepackage[pagecolor=none]{pagecolor}


% Set the titlepage font families







% Set the coverpage font families

\usepackage{bookmark}
\IfFileExists{xurl.sty}{\usepackage{xurl}}{} % add URL line breaks if available
\urlstyle{same}
\hypersetup{
  pdftitle={Civilian Impact of U.S. Drone vs.~Non-Drone Strikes in Somalia and Yemen},
  pdfauthor={Shanmei Wanyan; Daniel Dai; Keivan Bolouri; Itaru Fukushima; Linxue Guo; Evelyn Isaka},
  hidelinks,
  pdfcreator={LaTeX via pandoc}}


\title{Civilian Impact of U.S. Drone vs.~Non-Drone Strikes in Somalia
and Yemen}
\usepackage{etoolbox}
\makeatletter
\providecommand{\subtitle}[1]{% add subtitle to \maketitle
  \apptocmd{\@title}{\par {\large #1 \par}}{}{}
}
\makeatother
\subtitle{Assessing Humanitarian Impact with Count Regression Models.}
\author{Shanmei Wanyan \and Daniel Dai \and Keivan Bolouri \and Itaru
Fukushima \and Linxue Guo \and Evelyn Isaka}
\date{Invalid Date}
\begin{document}
%%%%% begin titlepage extension code

  \begin{frontmatter}

\begin{titlepage}

%%% TITLE PAGE START

% Set up alignment commands
%Page
\newcommand{\titlepagepagealign}{
\ifthenelse{\equal{left}{right}}{\raggedleft}{}
\ifthenelse{\equal{left}{center}}{\centering}{}
\ifthenelse{\equal{left}{left}}{\raggedright}{}
}


\newcommand{\titleandsubtitle}{
% Title and subtitle
{{\Large{\nohyphens{Civilian Impact of U.S. Drone vs.~Non-Drone Strikes
in Somalia and Yemen}}}\par
}%

\vspace{\betweentitlesubtitle}
{
{\textit{\nohyphens{Assessing Humanitarian Impact with Count Regression
Models.}}}\par
}}
\newcommand{\titlepagetitleblock}{
\titleandsubtitle
}

\newcommand{\authorstyle}[1]{{#1}}

\newcommand{\affiliationstyle}[1]{{#1}}

\newcommand{\titlepageauthorblock}{
{\authorstyle{\nohyphens{Shanmei
Wanyan}{\textsuperscript{1}},  \nohyphens{Daniel
Dai}{\textsuperscript{1}},  \nohyphens{Keivan
Bolouri}{\textsuperscript{1}},  \nohyphens{Itaru
Fukushima}{\textsuperscript{1}},  \nohyphens{Linxue
Guo}{\textsuperscript{1}} and \nohyphens{Evelyn
Isaka}{\textsuperscript{1}}}}}

\newcommand{\titlepageaffiliationblock}{
\hangindent=1em
\hangafter=1
{\affiliationstyle{
{1}.~University of California, Los Angeles,~Department of Statistics and
Data Science,~8125 Math Sciences Bldg, Los Angeles, CA 90095


\vspace{1\baselineskip} 
}}
}
\newcommand{\headerstyled}{%
{}
}
\newcommand{\footerstyled}{%
{STATS 140XP -- Practice of Statistical Consulting, Fall 2025
\url{https://keivanbolouri.github.io/finalProject140XP/}\\}
}
\newcommand{\datestyled}{%
{Invalid Date}
}


\newcommand{\titlepageheaderblock}{\headerstyled}

\newcommand{\titlepagefooterblock}{
\footerstyled
}

\newcommand{\titlepagedateblock}{
\datestyled
}

%set up blocks so user can specify order
\newcommand{\titleblock}{\newlength{\betweentitlesubtitle}
\setlength{\betweentitlesubtitle}{1pt}
{

{\titlepagetitleblock}
}

\vspace{4\baselineskip}
}

\newcommand{\authorblock}{{\titlepageauthorblock}

\vspace{2\baselineskip}
}

\newcommand{\affiliationblock}{{\titlepageaffiliationblock}

\vspace{2\baselineskip}
}

\newcommand{\logoblock}{{\includegraphics[width=0.1\textheight]{img/logo.png}}

\vspace{1\baselineskip}
}

\newcommand{\footerblock}{{\titlepagefooterblock}

\vspace{1pt}
}

\newcommand{\dateblock}{{\titlepagedateblock}

\vspace{0pt}
}

\newcommand{\headerblock}{}

\thispagestyle{empty} % no page numbers on titlepages


\newlength{\minipagewidth}
\setlength{\minipagewidth}{\textwidth}
\raggedright % single minipage
% [position of box][box height][inner position]{width}
% [s] means stretch out vertically; assuming there is a vfill
\begin{minipage}[b][\textheight][s]{\minipagewidth}
\titlepagepagealign
\headerblock

\titleblock

\authorblock

\affiliationblock

\vfill

\logoblock

\footerblock
\par

\end{minipage}\ifthenelse{\equal{}{right} \OR \equal{}{leftright} }{
\hspace{\B}
\vrulecode}{}
\clearpage
%%% TITLE PAGE END
\end{titlepage}
\setcounter{page}{1}
\end{frontmatter}

%%%%% end titlepage extension code

\renewcommand*\contentsname{Table of contents}
{
\setcounter{tocdepth}{2}
\tableofcontents
}
\listoffigures
\listoftables

\mainmatter
\chapter{Abstract}\label{abstract}

Since 2002, the United States has conducted largely hidden
counterterrorism campaigns in countries such as Somalia and Yemen,
raising ongoing concerns about their humanitarian impact. This project
asks how the characteristics and civilian costs of U.S. strikes differ
between these two theaters of war. Using open-source strike records
compiled by independent monitoring organizations, including the Bureau
of Investigative Journalism, We construct a combined dataset of U.S.
actions in Somalia and Yemen and analyze casualty patterns with negative
binomial regression. The analysis tests three hypotheses: whether
civilian casualty rates differ by country, whether drone strikes have
different effects across countries, and whether reporting uncertainty
varies between regions. The results show that, controlling for strike
characteristics and total fatalities, strikes in Yemen are associated
with nearly five times the civilian casualties of strikes in Somalia. By
contrast, there is no evidence that the impact of drone strikes on
civilian harm differs between the two countries, nor that overall
reporting uncertainty is systematically higher in one region than the
other. However, uncertainty is greater for drone and unconfirmed strikes
and lower when U.S. involvement is confirmed. These findings underscore
the unequal humanitarian burdens across theaters of U.S.
counterterrorism and highlight the need for more transparent and
consistent casualty reporting.

\chapter{Introduction}\label{introduction}

\section{}\label{section}

Since 2002, the United States has waged a largely clandestine drone war
in countries such as Yemen and Somalia, often far from public scrutiny
{[}1{]}. Although these counterterrorism strikes aim to eliminate
militant targets while minimizing risk to U.S. personnel, their
humanitarian consequences remain a pressing concern. The cost to
civilian life can be substantial. For example, an investigation found
that roughly one-third of those killed by U.S. drone strikes in Yemen in
2018 were likely civilians or pro-government allies {[}2{]}.

This research addresses a central question: \textbf{Do the
characteristics and human costs of U.S. counterterrorism strikes differ
between Somalia and Yemen---and if so, how?} Understanding such
differences is important both theoretically and practically.\\
Theoretically, comparing two distinct drone theaters can reveal how
local conditions---such as insurgent behavior, intelligence quality, and
terrain---shape strike outcomes. Practically, identifying where drone
operations are less effective in sparing civilians can guide
improvements in targeting procedures, transparency, and accountability
mechanisms.

These operations are often carried out ``out of sight,'' yet their
humanitarian consequences are very real {[}1{]}. Official reporting has
historically underestimated civilian casualties, prompting independent
organizations to investigate and publish alternative estimates {[}2{]}.
For instance, the U.S. government once claimed only 64--116 civilian
deaths from drone strikes outside declared warzones between 2009 and
2015, whereas independent monitors estimated several times more {[}2{]}.

In response to these discrepancies, numerous efforts have emerged to
document the drone war's toll. Pitch Interactive's \emph{Out of Sight,
Out of Mind} visualization cataloged CIA drone strikes and casualties in
Pakistan {[}1{]}. The Economist released infographics demonstrating
large gaps between official and independent casualty estimates. UCLA's
\textbf{Drone Wars} project created a cross-country dataset covering
Afghanistan, Pakistan, Somalia, and Yemen using records from the Bureau
of Investigative Journalism (BIJ) {[}3,4{]}.

These initiatives highlight the need for \textbf{rigorous, comparative
analysis}. Yet no study has systematically compared Somalia and Yemen
with respect to strike characteristics and humanitarian outcomes. This
paper fills that gap by leveraging detailed open-source strike records
from both countries to quantitatively assess differences in civilian
harm.

We explicitly test three hypotheses:

\begin{enumerate}
\def\labelenumi{\arabic{enumi}.}
\item
  \textbf{Hypothesis 1 -- Civilian Harm Difference:}\\
  Somalia and Yemen differ in their civilian casualty rates.
\item
  \textbf{Hypothesis 2 -- Drone Effectiveness Across Countries:}\\
  The effect of drone strikes on civilian casualties differs between
  Somalia and Yemen.
\item
  \textbf{Hypothesis 3 -- Reporting Uncertainty:}\\
  Casualty reporting uncertainty differs between regions.
\end{enumerate}

To evaluate these hypotheses, we construct a comprehensive dataset of
U.S. counterterrorism strikes in Somalia and Yemen from independent
monitoring organizations such as BIJ {[}1{]}. Because fatality reporting
is often uncertain, we use minimum--maximum casualty ranges {[}1{]}.
Civilian casualty counts exhibit strong overdispersion, so we employ
negative binomial regression to estimate the effects of region and
strike characteristics. This modeling framework allows us to determine
whether ``country'' remains a significant predictor of civilian harm
once contextual factors are controlled for.

\chapter{Literature Review}\label{literature-review}

Researchers and monitoring groups have spent many years examining how
many people are killed in U.S. drone strikes, but most work focuses on
one country at a time rather than comparing Somalia and Yemen directly.

Columbia Law School's Human Rights Clinic, in \emph{Counting Drone
Strike Deaths}, shows that official U.S. numbers often underestimate
civilian deaths. They recommend using casualty ranges (minimum--maximum)
because information from the ground is often unclear \textbf{{[}3{]}}.

The Bureau of Investigative Journalism (BIJ) collected open-source
reports for every known strike in Yemen, Somalia, Pakistan, and
Afghanistan. Their database records both minimum and maximum death
counts and distinguishes civilians from militants when possible, noting
that reports are often uncertain or contradictory \textbf{{[}5{]}}.

New America's \emph{Counterterrorism Wars} project compiles strike data
from Yemen and Somalia, listing total strikes and casualty ranges and
explaining how they classify victims when reports are vague or disputed
\textbf{{[}6{]}}.

Together, these sources show that:

\begin{enumerate}
\def\labelenumi{\arabic{enumi}.}
\item
  Independent groups usually find \textbf{more civilian deaths} than
  official U.S. reports.
\item
  Although detailed data exist for Yemen and Somalia, most previous
  analyses summarize each country separately rather than compare them
  statistically.
\end{enumerate}

Our study fits into this work by using open-source strike records to
conduct a direct, quantitative comparison between Somalia and Yemen.
Using negative binomial regression, we test whether the countries differ
in civilian casualty rates and the uncertainty of reported casualties,
controlling for strike characteristics.

\chapter{Data Processing}\label{data-processing}

\subsection{Data Import and Cleaning}\label{data-import-and-cleaning}

The dataset combines information on U.S. counterterrorism strikes in
\textbf{Somalia} and \textbf{Yemen}, spanning Somalia (2007--present)
and Yemen (2002--present). These data were originally compiled by
investigative journalists tracking U.S. drone and air strikes, including
casualties. In particular, the source appears to be the \textbf{Bureau
of Investigative Journalism (TBIJ)}, which maintains detailed records of
U.S. strikes in those
countries\href{https://www.defensepriorities.org/explainers/end-us-military-support-for-the-saudi-led-war-in-yemen/\#:~:text=peaked\%20in\%202017\%2C\%20with\%20more,war\%2Fyemen}{{[}7{]}}.
Each country's data was provided in a separate Excel worksheet (titled
\emph{``All US actions''} for Somalia and Yemen respectively),
containing reported strike dates, locations, strike types, and casualty
counts (with minimum and maximum estimates).

\subsection{Data Import and Cleaning}\label{data-import-and-cleaning-1}

\hfill\break
We imported two Excel datasets using \texttt{read\_excel()} in R:

\textbf{Somalia:} \texttt{us-strikes-in-somalia-2007-to-present.xls}

\textbf{Yemen:} \texttt{us-strikes-in-yemen-2002-to-present.xlsx}\\
Each file was read from the ``All US actions'' sheet into
\texttt{somalia\_raw} and \texttt{yemen\_raw}.

Using \texttt{dplyr::transmute()}, we extracted and standardized key
variables to match across datasets. This included:

Assigning a \texttt{region} label (Somalia or Yemen)

\begin{itemize}
\item
  Converting \texttt{Date} fields to proper date format

  Creating indicators for drone strikes (\texttt{drone}) and U.S.
  confirmation (\texttt{us\_confirmed})

  Harmonizing strike and casualty counts (\texttt{min\_}/\texttt{max\_}
  values for killed, civilians, children, and injured)
\end{itemize}

These transformations produced two cleaned data frames with identical
structures, enabling easy merging.

\subsection{Combining and Preparing the
Dataset}\label{combining-and-preparing-the-dataset}

To support hypothesis testing, we created three key analytical variables
from the cleaned and combined dataset. \texttt{civilian\_casualties}
represents the minimum number of civilians killed per strike, using the
\texttt{min\_civilians} field as a conservative estimate.
\texttt{total\_killed} captures the minimum total fatalities
(\texttt{min\_killed}) for each incident, providing a standardized
baseline for analysis. \texttt{uncertainty\_killed} quantifies reporting
uncertainty by calculating the difference between \texttt{max\_killed}
and \texttt{min\_killed}. These derived variables help assess both the
scale of violence and the variability in casualty reporting across
strikes and regions. All three were added to the dataset and are ready
for descriptive analysis.

\begin{longtable}[]{@{}
  >{\raggedright\arraybackslash}p{(\linewidth - 6\tabcolsep) * \real{0.1026}}
  >{\raggedleft\arraybackslash}p{(\linewidth - 6\tabcolsep) * \real{0.3205}}
  >{\raggedleft\arraybackslash}p{(\linewidth - 6\tabcolsep) * \real{0.2308}}
  >{\raggedleft\arraybackslash}p{(\linewidth - 6\tabcolsep) * \real{0.3462}}@{}}
\caption{Summary of Key Derived Variables by Region}\tabularnewline
\toprule\noalign{}
\begin{minipage}[b]{\linewidth}\raggedright
Region
\end{minipage} & \begin{minipage}[b]{\linewidth}\raggedleft
Avg. Civilian Casualties
\end{minipage} & \begin{minipage}[b]{\linewidth}\raggedleft
Avg. Total Killed
\end{minipage} & \begin{minipage}[b]{\linewidth}\raggedleft
Avg. Reporting Uncertainty
\end{minipage} \\
\midrule\noalign{}
\endfirsthead
\toprule\noalign{}
\begin{minipage}[b]{\linewidth}\raggedright
Region
\end{minipage} & \begin{minipage}[b]{\linewidth}\raggedleft
Avg. Civilian Casualties
\end{minipage} & \begin{minipage}[b]{\linewidth}\raggedleft
Avg. Total Killed
\end{minipage} & \begin{minipage}[b]{\linewidth}\raggedleft
Avg. Reporting Uncertainty
\end{minipage} \\
\midrule\noalign{}
\endhead
\bottomrule\noalign{}
\endlastfoot
Somalia & 0.08 & 6.54 & 1.56 \\
Yemen & 0.64 & 4.67 & 1.84 \\
\end{longtable}

\subsection{Finalizing the Dataset for
Analysis}\label{finalizing-the-dataset-for-analysis}

The last step shown is the creation of \texttt{combined\_model}, which
is a filtered version of the data ready for modeling or statistical
analysis. Here we \textbf{restrict to complete cases} for the key
outcome variables of interest. The code
\texttt{filter(!is.na(civilian\_casualties),\ !is.na(uncertainty\_killed),\ !is.na(min\_strikes),\ !is.na(total\_killed))}
removes any strikes that still have missing values in those crucial
fields.

In practice, because we already replaced NAs with 0 or other values for
most of these, there may be very few records dropped. However, this
filter is a safety measure to ensure that the modeling dataset doesn't
include any undefined values. For example, if a particular entry had an
entirely missing civilian casualty field in the raw data (and somehow
our earlier replacement didn't catch it), or if any event lacks data on
number of strikes or total killed, it will be excluded. The end result
\texttt{combined\_model} contains only strikes with valid
\texttt{civilian\_casualties}, \texttt{total\_killed},
\texttt{uncertainty\_killed}, and \texttt{min\_strikes} values. This
will be the dataset used in subsequent analysis and hypothesis testing.

\chapter{Methods}\label{methods}

\section{Statistical Methods}\label{statistical-methods}

\subsection{Hypothesis Tests}\label{hypothesis-tests}

\subsection{\texorpdfstring{\textbf{Hypothesis 1: Civilian Harm
Difference}}{Hypothesis 1: Civilian Harm Difference}}\label{hypothesis-1-civilian-harm-difference}

\[
\begin{aligned}
H_{0}: &\ \text{Drone strikes have the same civilian impact in Somalia and Yemen.} \\
H_{1}: &\ \text{Drone strikes have different civilian impacts across the two regions.}
\end{aligned}
\]

To test whether Somalia and Yemen differ in civilian casualty outcomes,
we estimate the model:

\[
\begin{aligned}
E(\text{civilian casualties}) = {} &
\beta_0 
+ \beta_1(\text{region}) 
+ \beta_2(\text{drone}) 
+ \beta_3(\text{US confirmed}) 
+ \beta_4(\text{minimum strikes}) \\
& + \beta_5(\text{total killed})
\end{aligned}
\]

\begin{longtable}[]{@{}
  >{\raggedright\arraybackslash}p{(\linewidth - 2\tabcolsep) * \real{0.2222}}
  >{\raggedright\arraybackslash}p{(\linewidth - 2\tabcolsep) * \real{0.7778}}@{}}
\caption{Variable Definitions}\tabularnewline
\toprule\noalign{}
\begin{minipage}[b]{\linewidth}\raggedright
Variable
\end{minipage} & \begin{minipage}[b]{\linewidth}\raggedright
Description
\end{minipage} \\
\midrule\noalign{}
\endfirsthead
\toprule\noalign{}
\begin{minipage}[b]{\linewidth}\raggedright
Variable
\end{minipage} & \begin{minipage}[b]{\linewidth}\raggedright
Description
\end{minipage} \\
\midrule\noalign{}
\endhead
\bottomrule\noalign{}
\endlastfoot
Civilian casualties & Number of civilians reported killed in the strike
(outcome variable). \\
Region & Country where the strike occurred (Somalia or Yemen). \\
Drone & Indicates whether the strike was carried out by a drone (1 =
drone). \\
US confirmed & Whether the strike was officially confirmed by the U.S.
government. \\
Minimum strikes & Minimum number of strike events associated with the
record. \\
Total killed & Minimum number of total fatalities (civilians +
militants). \\
\end{longtable}

\subsection{\texorpdfstring{\textbf{Hypothesis 2: Drone Effectiveness
Across
Countries}}{Hypothesis 2: Drone Effectiveness Across Countries}}\label{hypothesis-2-drone-effectiveness-across-countries}

\[
\begin{aligned}
H_{0}: &\ \text{Drone use affects civilian casualties in the same way in both Somalia and Yemen.} \\
H_{1}: &\ \text{Drone use affects civilian casualties differently across Somalia and Yemen.}
\end{aligned}
\]

To evaluate whether the civilian impact of drone strikes varies by
region, we include an interaction term between drone use and region:

\[
\begin{aligned}
E(\text{civilian casualties}) = {} &
\beta_0 
+ \beta_1(\text{drone})
+ \beta_2(\text{region}) 
+ \beta_3(\text{drone} \times \text{region}) \\
& + \beta_4(\text{US confirmed})
+ \beta_5(\text{minimum strikes})
+ \beta_6(\text{total killed})
\end{aligned}
\]

\subsection{\texorpdfstring{\textbf{Hypothesis 3: Reporting Uncertainty
Difference}}{Hypothesis 3: Reporting Uncertainty Difference}}\label{hypothesis-3-reporting-uncertainty-difference}

\[
\begin{aligned}
H_{0}: &\ \text{Reporting uncertainty does not differ between Somalia and Yemen.} \\
H_{1}: &\ \text{Reporting uncertainty differs between Somalia and Yemen.}
\end{aligned}
\]

To assess whether casualty reporting uncertainty differs between
regions, we model the uncertainty metric. We modeled casualty reporting
uncertainty (defined as \texttt{max\_killed\ -\ min\_killed}) region and
strike characteristics as predictors.

\[
E(\text{Uncertainty in casualties}) =
\beta_0
+ \beta_1\,\text{Region}
+ \beta_2\,\text{Drone}
+ \beta_3\,\text{US confirmed}
+ \beta_4\,\text{Minimum strikes}
\]

where:

\[
\text{Uncertainty in casualties} = 
\text{Maximum killed} - \text{Minimum killed}
\]

\chapter{Model Selection}\label{model-selection}

Because our prediction variable is a count---specifically, the number of
civilians killed in each strike---we use statistical models designed for
count data. A natural starting point is the \textbf{Poisson regression},
which assumes that the mean and variance of the outcome are equal
\(E(x) = \mathrm{Var}(x)\). However, in our dataset the variance is much
larger than the mean, a condition known as \textbf{overdispersion}. When
overdispersion is present, Poisson regression underestimates the true
variability and produces misleadingly small standard errors. To address
this, we use a \textbf{negative binomial regression}, which adds a
dispersion parameter that allows the variance to exceed the mean. This
makes the negative binomial model much better suited for modeling
drone-strike casualty counts and provides more reliable estimates of how
factors such as region, drone use, and confirmation status relate to
civilian harm.

\begin{Shaded}
\begin{Highlighting}[]
\NormalTok{mean\_civ }\OtherTok{\textless{}{-}} \FunctionTok{mean}\NormalTok{(combined\_model}\SpecialCharTok{$}\NormalTok{civilian\_casualties)}
\NormalTok{var\_civ  }\OtherTok{\textless{}{-}} \FunctionTok{var}\NormalTok{(combined\_model}\SpecialCharTok{$}\NormalTok{civilian\_casualties)}

\FunctionTok{c}\NormalTok{(}\AttributeTok{mean =}\NormalTok{ mean\_civ, }\AttributeTok{variance =}\NormalTok{ var\_civ)}
\end{Highlighting}
\end{Shaded}

\begin{verbatim}
     mean  variance 
0.4338521 8.0043879 
\end{verbatim}

Showt that our data is overdispersion: \(E(x) < \mathrm{Var}(x)\)

In our combined Somalia--Yemen dataset, civilian casualties have a mean
of \textbf{0.43} and a variance of \textbf{8.00}, so the variance is
about \textbf{18 times} larger than the mean. This large
variance-to-mean ratio indicates substantial overdispepersion.

\hfill\break
To verify whether a Poisson model was appropriate for our outcome
variable, we formally tested for overdispersion. We first fit a Poisson
regression using civilian casualties as the count outcome and calculated
the dispersion statistic by dividing the residual deviance by the
residual degrees of freedom.

\subsection{Poisson dispersion test}\label{poisson-dispersion-test}

\[
\begin{aligned}
H_0 &: \text{dispersion} = 1 \quad (\text{Poisson adequate})\\
H_a &: \text{dispersion} > 1 \quad (\text{overdispersion})
\end{aligned}
\]

\begin{Shaded}
\begin{Highlighting}[]
\FunctionTok{library}\NormalTok{(MASS)}
\FunctionTok{library}\NormalTok{(AER)   }\CommentTok{\# for dispersiontest}

\CommentTok{\# Poisson version of H1 model}
\NormalTok{pois\_h1 }\OtherTok{\textless{}{-}} \FunctionTok{glm}\NormalTok{(}
\NormalTok{  civilian\_casualties }\SpecialCharTok{\textasciitilde{}}\NormalTok{ region }\SpecialCharTok{+}\NormalTok{ drone }\SpecialCharTok{+}\NormalTok{ us\_confirmed}
  \SpecialCharTok{+}\NormalTok{ min\_strikes }\SpecialCharTok{+}\NormalTok{ total\_killed,}
  \AttributeTok{family =} \FunctionTok{poisson}\NormalTok{(}\AttributeTok{link =} \StringTok{"log"}\NormalTok{),}
  \AttributeTok{data =}\NormalTok{ combined\_model}
\NormalTok{)}

\CommentTok{\# 3a. Quick dispersion estimate: residual deviance / df}
\NormalTok{dispersion\_est }\OtherTok{\textless{}{-}}\NormalTok{ pois\_h1}\SpecialCharTok{$}\NormalTok{deviance }\SpecialCharTok{/}\NormalTok{ pois\_h1}\SpecialCharTok{$}\NormalTok{df.residual}
\NormalTok{dispersion\_est}
\end{Highlighting}
\end{Shaded}

\begin{verbatim}
[1] 1.945971
\end{verbatim}

\begin{Shaded}
\begin{Highlighting}[]
\CommentTok{\# 3b. Formal test}
\FunctionTok{dispersiontest}\NormalTok{(pois\_h1)}
\end{Highlighting}
\end{Shaded}

\begin{verbatim}

    Overdispersion test

data:  pois_h1
z = 2.3989, p-value = 0.008222
alternative hypothesis: true dispersion is greater than 1
sample estimates:
dispersion 
  11.36617 
\end{verbatim}

The resulting value of approximately \textbf{1.95} already suggested
that the variance in the data was nearly twice as large as the Poisson
model allows. We then conducted a formal \textbf{overdispersion test}
using \texttt{dispersiontest()} from the \emph{AER} package. The test
returned a z-value of \textbf{2.40} with a p-value of \textbf{0.008},
indicating statistically significant overdispersion. In other words, the
Poisson assumption that the mean equals the variance is violated.
Because the data exhibit much greater variability than the Poisson model
can accommodate, this test confirms that a \textbf{negative binomial
regression}---which includes an additional.

Consequently, we use negative binomial regression, which relaxes the
equidispersion assumption and is more appropriate for these data.

\chapter{Statistical Testing}\label{statistical-testing}

\textbf{Hypothesis Test 1}

\begin{Shaded}
\begin{Highlighting}[]
\FunctionTok{library}\NormalTok{(dplyr)}
\FunctionTok{library}\NormalTok{(stringr)}
\FunctionTok{library}\NormalTok{(MASS)}
\FunctionTok{library}\NormalTok{(broom)}



\NormalTok{model\_h1 }\OtherTok{\textless{}{-}} \FunctionTok{glm.nb}\NormalTok{(}
\NormalTok{  civilian\_casualties }\SpecialCharTok{\textasciitilde{}}\NormalTok{ region }
  \SpecialCharTok{+}\NormalTok{ drone }\SpecialCharTok{+}\NormalTok{ us\_confirmed }\SpecialCharTok{+}\NormalTok{ min\_strikes }\SpecialCharTok{+}\NormalTok{ total\_killed,}
  \AttributeTok{data =}\NormalTok{ combined\_model}
\NormalTok{)}

\FunctionTok{summary}\NormalTok{(model\_h1)}
\end{Highlighting}
\end{Shaded}

\begin{verbatim}

Call:
glm.nb(formula = civilian_casualties ~ region + drone + us_confirmed + 
    min_strikes + total_killed, data = combined_model, init.theta = 0.04945140151, 
    link = log)

Coefficients:
             Estimate Std. Error z value Pr(>|z|)    
(Intercept)  -2.08959    1.02564  -2.037   0.0416 *  
regionYemen   1.59150    0.64403   2.471   0.0135 *  
drone        -0.00298    0.58431  -0.005   0.9959    
us_confirmed -0.19350    0.56118  -0.345   0.7302    
min_strikes  -1.02747    0.68030  -1.510   0.1310    
total_killed  0.14571    0.01948   7.481 7.39e-14 ***
---
Signif. codes:  0 '***' 0.001 '**' 0.01 '*' 0.05 '.' 0.1 ' ' 1

(Dispersion parameter for Negative Binomial(0.0495) family taken to be 1)

    Null deviance: 143.44  on 513  degrees of freedom
Residual deviance: 107.17  on 508  degrees of freedom
AIC: 469.8

Number of Fisher Scoring iterations: 1

              Theta:  0.0495 
          Std. Err.:  0.0102 
Warning while fitting theta: alternation limit reached 

 2 x log-likelihood:  -455.8010 
\end{verbatim}

\begin{Shaded}
\begin{Highlighting}[]
\FunctionTok{tidy}\NormalTok{(model\_h1, }\AttributeTok{exponentiate =} \ConstantTok{TRUE}\NormalTok{, }\AttributeTok{conf.int =} \ConstantTok{TRUE}\NormalTok{)  }
\end{Highlighting}
\end{Shaded}

\begin{verbatim}
# A tibble: 6 × 7
  term         estimate std.error statistic  p.value conf.low conf.high
  <chr>           <dbl>     <dbl>     <dbl>    <dbl>    <dbl>     <dbl>
1 (Intercept)     0.124    1.03    -2.04    4.16e- 2   0.0185      1.21
2 regionYemen     4.91     0.644    2.47    1.35e- 2   1.57       15.6 
3 drone           0.997    0.584   -0.00510 9.96e- 1   0.326       2.80
4 us_confirmed    0.824    0.561   -0.345   7.30e- 1   0.246       2.40
5 min_strikes     0.358    0.680   -1.51    1.31e- 1   0.0657      1.03
6 total_killed    1.16     0.0195   7.48    7.39e-14  NA          NA   
\end{verbatim}

\subsection{Rsult for Hypothesis 1: Civilian Harm
Differences}\label{rsult-for-hypothesis-1-civilian-harm-differences}

Strikes in \textbf{Yemen} show significantly higher civilian casualties
than those in Somalia.\\
The coefficient for Yemen is \textbf{1.59} (\emph{p} = 0.0135),
corresponding to an IRR of \textbf{4.9}, meaning Yemen strikes produce
nearly \textbf{5×} the civilian casualties of Somalia.\\
Total fatalities are also positively associated with civilian casualties
(coef = \textbf{0.146}, \emph{p} \textless{} 0.001).\\
\textbf{Conclusion:} Civilian harm is significantly higher in Yemen →
\emph{H1 supported}

\textbf{Hypothesis Test 2}

\begin{Shaded}
\begin{Highlighting}[]
\NormalTok{model\_h2 }\OtherTok{\textless{}{-}} \FunctionTok{glm.nb}\NormalTok{(}
\NormalTok{  civilian\_casualties }\SpecialCharTok{\textasciitilde{}}\NormalTok{ drone }\SpecialCharTok{*}\NormalTok{ region }\SpecialCharTok{+}
\NormalTok{    us\_confirmed }\SpecialCharTok{+}\NormalTok{ min\_strikes }\SpecialCharTok{+}\NormalTok{ total\_killed,}
  \AttributeTok{data =}\NormalTok{ combined\_model}
\NormalTok{)}

\FunctionTok{summary}\NormalTok{(model\_h2)}
\end{Highlighting}
\end{Shaded}

\begin{verbatim}

Call:
glm.nb(formula = civilian_casualties ~ drone * region + us_confirmed + 
    min_strikes + total_killed, data = combined_model, init.theta = 0.04963878986, 
    link = log)

Coefficients:
                  Estimate Std. Error z value Pr(>|z|)    
(Intercept)       -2.15732    1.03711  -2.080   0.0375 *  
drone              0.33091    1.00164   0.330   0.7411    
regionYemen        1.80725    0.83212   2.172   0.0299 *  
us_confirmed      -0.21220    0.55986  -0.379   0.7047    
min_strikes       -1.02119    0.66922  -1.526   0.1270    
total_killed       0.14319    0.01965   7.288 3.14e-13 ***
drone:regionYemen -0.49367    1.22774  -0.402   0.6876    
---
Signif. codes:  0 '***' 0.001 '**' 0.01 '*' 0.05 '.' 0.1 ' ' 1

(Dispersion parameter for Negative Binomial(0.0496) family taken to be 1)

    Null deviance: 143.81  on 513  degrees of freedom
Residual deviance: 107.26  on 507  degrees of freedom
AIC: 471.63

Number of Fisher Scoring iterations: 1

              Theta:  0.0496 
          Std. Err.:  0.0103 
Warning while fitting theta: alternation limit reached 

 2 x log-likelihood:  -455.6270 
\end{verbatim}

\begin{Shaded}
\begin{Highlighting}[]
\FunctionTok{tidy}\NormalTok{(model\_h2, }\AttributeTok{exponentiate =} \ConstantTok{TRUE}\NormalTok{, }\AttributeTok{conf.int =} \ConstantTok{TRUE}\NormalTok{)}
\end{Highlighting}
\end{Shaded}

\begin{verbatim}
# A tibble: 7 × 7
  term              estimate std.error statistic  p.value conf.low conf.high
  <chr>                <dbl>     <dbl>     <dbl>    <dbl>    <dbl>     <dbl>
1 (Intercept)          0.116    1.04      -2.08  3.75e- 2   0.0171      1.14
2 drone                1.39     1.00       0.330 7.41e- 1   0.215      11.5 
3 regionYemen          6.09     0.832      2.17  2.99e- 2   1.38       32.8 
4 us_confirmed         0.809    0.560     -0.379 7.05e- 1   0.241       2.36
5 min_strikes          0.360    0.669     -1.53  1.27e- 1   0.0671      1.02
6 total_killed         1.15     0.0196     7.29  3.14e-13  NA          NA   
7 drone:regionYemen    0.610    1.23      -0.402 6.88e- 1   0.0500      5.87
\end{verbatim}

\subsection{Result for Hypothesis 2: Drone Effectiveness by
Country}\label{result-for-hypothesis-2-drone-effectiveness-by-country}

The key interaction term \textbf{drone × region (Yemen)} is \textbf{not
significant} (coef = −0.49, \emph{p} = 0.688).\\
Drone use alone is also not significant (coef = 0.33, \emph{p} =
0.741).\\
\textbf{Conclusion:} Drones do not affect civilian casualties
differently across countries → \emph{H2 not supported}.

\textbf{Hypothesis Test 3}

\begin{Shaded}
\begin{Highlighting}[]
\NormalTok{model\_h3 }\OtherTok{\textless{}{-}} \FunctionTok{glm.nb}\NormalTok{(}
\NormalTok{  uncertainty\_killed }\SpecialCharTok{\textasciitilde{}}\NormalTok{ region }\SpecialCharTok{+} 
\NormalTok{    drone }\SpecialCharTok{+}\NormalTok{ us\_confirmed }\SpecialCharTok{+}\NormalTok{ min\_strikes,}
  \AttributeTok{data =}\NormalTok{ combined\_model}
\NormalTok{)}
\FunctionTok{summary}\NormalTok{(model\_h3)}
\end{Highlighting}
\end{Shaded}

\begin{verbatim}

Call:
glm.nb(formula = uncertainty_killed ~ region + drone + us_confirmed + 
    min_strikes, data = combined_model, init.theta = 0.2157180244, 
    link = log)

Coefficients:
             Estimate Std. Error z value Pr(>|z|)   
(Intercept)   0.64763    0.30943   2.093  0.03635 * 
regionYemen  -0.16294    0.26446  -0.616  0.53781   
drone         0.61827    0.24966   2.476  0.01327 * 
us_confirmed -0.69343    0.25255  -2.746  0.00604 **
min_strikes   0.08520    0.05284   1.612  0.10687   
---
Signif. codes:  0 '***' 0.001 '**' 0.01 '*' 0.05 '.' 0.1 ' ' 1

(Dispersion parameter for Negative Binomial(0.2157) family taken to be 1)

    Null deviance: 400.69  on 513  degrees of freedom
Residual deviance: 385.48  on 509  degrees of freedom
AIC: 1571.1

Number of Fisher Scoring iterations: 1

              Theta:  0.2157 
          Std. Err.:  0.0218 

 2 x log-likelihood:  -1559.1480 
\end{verbatim}

\begin{Shaded}
\begin{Highlighting}[]
\FunctionTok{tidy}\NormalTok{(model\_h3, }\AttributeTok{exponentiate =} \ConstantTok{TRUE}\NormalTok{, }\AttributeTok{conf.int =} \ConstantTok{TRUE}\NormalTok{)}
\end{Highlighting}
\end{Shaded}

\begin{verbatim}
# A tibble: 5 × 7
  term         estimate std.error statistic p.value conf.low conf.high
  <chr>           <dbl>     <dbl>     <dbl>   <dbl>    <dbl>     <dbl>
1 (Intercept)     1.91     0.309      2.09  0.0363     1.13      3.34 
2 regionYemen     0.850    0.264     -0.616 0.538      0.536     1.33 
3 drone           1.86     0.250      2.48  0.0133     1.18      2.91 
4 us_confirmed    0.500    0.253     -2.75  0.00604    0.305     0.793
5 min_strikes     1.09     0.0528     1.61  0.107      0.990     1.28 
\end{verbatim}

\begin{Shaded}
\begin{Highlighting}[]
\NormalTok{combined\_model }\SpecialCharTok{\%\textgreater{}\%}
  \FunctionTok{group\_by}\NormalTok{(region, drone) }\SpecialCharTok{\%\textgreater{}\%}
  \FunctionTok{summarise}\NormalTok{(}
    \AttributeTok{mean\_civilian\_casualties =} \FunctionTok{mean}\NormalTok{(civilian\_casualties, }\AttributeTok{na.rm =} \ConstantTok{TRUE}\NormalTok{),}
    \AttributeTok{n =} \FunctionTok{n}\NormalTok{(),}
    \AttributeTok{.groups =} \StringTok{"drop"}
\NormalTok{  )}
\end{Highlighting}
\end{Shaded}

\begin{verbatim}
# A tibble: 4 × 4
  region  drone mean_civilian_casualties     n
  <fct>   <int>                    <dbl> <int>
1 Somalia     0                   0.0753   146
2 Somalia     1                   0.114     44
3 Yemen       0                   1.38      76
4 Yemen       1                   0.411    248
\end{verbatim}

\begin{Shaded}
\begin{Highlighting}[]
\NormalTok{combined\_model }\SpecialCharTok{\%\textgreater{}\%}
  \FunctionTok{group\_by}\NormalTok{(region) }\SpecialCharTok{\%\textgreater{}\%}
  \FunctionTok{summarise}\NormalTok{(}
    \AttributeTok{mean\_uncertainty =} \FunctionTok{mean}\NormalTok{(uncertainty\_killed, }\AttributeTok{na.rm =} \ConstantTok{TRUE}\NormalTok{),}
    \AttributeTok{n =} \FunctionTok{n}\NormalTok{(),}
    \AttributeTok{.groups =} \StringTok{"drop"}
\NormalTok{  )}
\end{Highlighting}
\end{Shaded}

\begin{verbatim}
# A tibble: 2 × 3
  region  mean_uncertainty     n
  <fct>              <dbl> <int>
1 Somalia             1.56   190
2 Yemen               1.84   324
\end{verbatim}

\subsection{Result for Hypothesis 3: Reporting
Uncertainty}\label{result-for-hypothesis-3-reporting-uncertainty}

Reporting uncertainty does \textbf{not} differ between regions (Yemen
coef = −0.16, \emph{p} = 0.538).\\
However, drone strikes show \textbf{higher uncertainty} (coef = 0.62,
IRR = \textbf{1.86}, \emph{p} = 0.013), while confirmed U.S. strikes
show \textbf{lower uncertainty} (coef = −0.69, IRR = \textbf{0.50},
\emph{p} = 0.006).\\
\textbf{Conclusion:} No regional difference in uncertainty → \emph{H3
not supported}, though uncertainty varies by strike type.

\chapter{Visualizations}\label{visualizations}

\section{Visualization (Test 1)}\label{visualization-test-1}

\[
\begin{aligned}
H_{0}: &\ \text{Drone strikes have the same civilian impact in Somalia and Yemen.} \\
H_{1}: &\ \text{Drone strikes have different civilian impacts across the two regions.}
\end{aligned}
\]

\[
\begin{aligned}
E(\text{civilian casualties}) = {} &
\beta_0 
+ \beta_1(\text{region}) 
+ \beta_2(\text{drone}) 
+ \beta_3(\text{US confirmed}) 
+ \beta_4(\text{minimum strikes}) \\
& + \beta_5(\text{total killed})
\end{aligned}
\]

\includegraphics[width=7in,height=\textheight,keepaspectratio]{Example_files/figure-html/unnamed-chunk-9-1.png}

\includegraphics[width=7in,height=\textheight,keepaspectratio]{Example_files/figure-html/unnamed-chunk-10-1.png}

\section{Visualization (Test 2)}\label{visualization-test-2}

\[
\begin{aligned}
H_{0}: &\ \text{Drone use affects civilian casualties in the same way in both Somalia and Yemen.} \\
H_{1}: &\ \text{Drone use affects civilian casualties differently across Somalia and Yemen.}
\end{aligned}
\]

\[
\begin{aligned}
E(\text{civilian casualties}) = {} &
\beta_0 
+ \beta_1(\text{drone})
+ \beta_2(\text{region}) 
+ \beta_3(\text{drone} \times \text{region}) \\
& + \beta_4(\text{US confirmed})
+ \beta_5(\text{minimum strikes})
+ \beta_6(\text{total killed})
\end{aligned}
\]

\includegraphics[width=7in,height=\textheight,keepaspectratio]{Example_files/figure-html/unnamed-chunk-11-1.png}

\includegraphics[width=7in,height=\textheight,keepaspectratio]{Example_files/figure-html/unnamed-chunk-12-1.png}

\section{Visualization (Test 3)}\label{visualization-test-3}

\[
\begin{aligned}
H_{0}: &\ \text{Reporting uncertainty does not differ between Somalia and Yemen.} \\
H_{1}: &\ \text{Reporting uncertainty differs between Somalia and Yemen.}
\end{aligned}
\]

\[
E(\text{Uncertainty in casualties}) =
\beta_0
+ \beta_1\,\text{Region}
+ \beta_2\,\text{Drone}
+ \beta_3\,\text{US confirmed}
+ \beta_4\,\text{Minimum strikes}
\]

\includegraphics[width=7in,height=\textheight,keepaspectratio]{Example_files/figure-html/unnamed-chunk-13-1.png}

\includegraphics[width=7in,height=\textheight,keepaspectratio]{Example_files/figure-html/unnamed-chunk-14-1.png}

\chapter{Conclusion}\label{conclusion}

Our analysis shows clear and meaningful differences in the humanitarian
impact of U.S. counterterrorism strikes in Somalia and Yemen. Using
negative binomial regression to account for overdispersed count data, we
find that strikes in Yemen are associated with nearly five times the
civilian casualties of those in Somalia, even after controlling for
drone use, confirmation status, and strike characteristics. Contrary to
expectations, drone strikes do not have significantly different effects
across the two countries, suggesting that broader regional factors---not
simply weapon type---shape civilian outcomes. We also find no regional
difference in reporting uncertainty, although uncertainty is higher for
drone and unconfirmed strikes. Together, these results highlight the
importance of transparent casualty reporting and the need to consider
local conflict conditions when evaluating the effectiveness and
humanitarian cost of U.S. strikes.

\chapter{Author Contributions}\label{author-contributions}

In this project, the team collaborated effectively by distributing key
responsibilities across members. Daniel Dai and Keivan Bolouri led the
design and preparation of the poster, while Evelyn Isaka and Shanmei
Wanyan delivered the main presentation. The written report was developed
by Shanmei Wanyan, Keivan Bolouri, and Linxue Guo, ensuring clear
documentation of the project's methods and results. Coding and
analytical implementation were carried out by Itaru Fukushima and Daniel
Dai, whose contributions supported the project's computational aspects.

\chapter{Acknowledgments}\label{acknowledgments}

We would like to extend our deepest gratitude to Professor
\textbf{Vivian Lew}, whose patience, dedication, and insightful guidance
were invaluable throughout every stage of this project. Her
encouragement and clear explanations helped us strengthen our
understanding and improve the quality of our work. We are also grateful
to our TA, \textbf{Jose Toledo Luna}, for his consistent support,
helpful feedback, and willingness to assist whenever we needed
clarification.

\chapter{References}\label{references}

\phantomsection\label{refs}

{[}1{]} Currier, Cora. ``Everything We Know So Far About Drone
Strikes.''~\emph{ProPublica}, 5 Feb.~2013.\\
{[}2{]} Woods, Chris. ``Why White House Civilian Casualty Figures Are a
Wild Underestimate.''~\emph{Bureau of Investigative Journalism}, 1 July
2016.\\
{[}3{]} Columbia Law School Human Rights Clinic.~\emph{Counting Drone
Strike Deaths}. Human Rights Institute, Columbia Law School,
Oct.~2012.\\
{[}4{]} Bergen, Peter, David Sterman, and Melissa Salyk-Virk. ``The
Drone War in Somalia.'' New America, 30 Mar.~2020.\\
{[}5{]} Bureau of Investigative Journalism. ``Our Methodology.'' BIJ.\\
{[}6{]} New America. ``America's Counterterrorism Wars: Methodology.''
Future Security Program.

\href{https://www.defensepriorities.org/explainers/end-us-military-support-for-the-saudi-led-war-in-yemen/\#:~:text=peaked\%20in\%202017\%2C\%20with\%20more,war\%2Fyemen}{{[}7{]}}
End U.S. military support for the Saudi-led war in Yemen - Defense
Priorities

\url{https://www.defensepriorities.org/explainers/end-us-military-support-for-the-saudi-led-war-in-yemen/}


\backmatter


\end{document}
